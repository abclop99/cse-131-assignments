\begin{document}
	\chapter*{Compiler 62}

	\section{Tracing the compiler}

	\subsection{Program 1}

	\begin{lstlisting}[title=Factorial Program, language=Lisp]
		(let
			((i 1) (acc 1))
			(loop
				(if (> i input)
					(break acc)
					(block
						(set! acc (* acc i))
						(set! i (+ i 1))
					)
				)
			)
		)
	\end{lstlisting}

	\subsubsection{Relevant snippets}

	\begin{enumerate}[\textrm{Snippet} 1:]
		\item A snippet of the part of \verb|parse_expr| that handles parsing to \verb|BinOp|.
			\begin{lstlisting}[language=Rust]
				[Sexp::Atom(S(op)), e1, e2] if op == "+" =>
				    Expr::BinOp(Op2::Plus, Box::new(parse_expr(e1)), Box::new(parse_expr(e2))),
				[Sexp::Atom(S(op)), e1, e2] if op == "-" =>
				    Expr::BinOp(Op2::Minus, Box::new(parse_expr(e1)), Box::new(parse_expr(e2))),
				[Sexp::Atom(S(op)), e1, e2] if op == "*" =>
				    Expr::BinOp(Op2::Times, Box::new(parse_expr(e1)), Box::new(parse_expr(e2))),
				. . . (other operators)
			\end{lstlisting}
			\begin{itemize}
				\item A lot of repetition in the code. You can match on the pattern \verb|[Sexp::Atom(S(op)), e1, e2]| once and then match on the operator. Just make sure you match any other expressions that could have the same format first:

				\begin{lstlisting}[language=Rust]
						[Sexp::Atom(S(op)), e1, e2] => {
						    let op = match op.as_str() {
						        "+" => Op2::Plus,
						        "-" => Op2::Minus,
						        "*" => Op2::Times,
							. . .
						        _ => panic!("Invalid operator"),
						    };
						    Expr::BinOp(op, Box::new(parse_expr(e1)), Box::new(parse_expr(e2)))
						}
					\end{lstlisting}
			\end{itemize}

		\pagebreak
		\item This snippet is from the \verb|instr_to_string| function.
			\begin{lstlisting}[language=Rust]
				enum Instr {
				    \ldots
				    IJmp(i32),
				    IJe(i32),
				    IJne(i32),
				    IJo(i32), // it's jover
				    ILabel(i32),
				}
				. . . ( other code )
				let label_name = |l: i32| {
				    let label_name = |l: i32| {
				        if l < 0 {
				            format!(".E{}", -l)
				        } else {
				            format!(".L{}", l)
				        }
				    };
				    . . .
				};
			\end{lstlisting}

			This snippet handles converting labels to strings for the \verb|IJmp|, \verb|IJe|, \verb|IJne|, \verb|IJo|, and \verb|ILabel| instructions.

			\begin{enumerate}
				\item The instructions should not all begin with the same prefix for no reason because it is useless to do so. \verb|cargo clippy| warns about this:
					\begin{lstlisting}[numbers=none, language=bash, basicstyle=\ttfamily\small, title=\texttt{cargo clippy} output]
						warning: all variants have the same prefix: `I`
						  --> src/main.rs:26:1
						   |
						26 | / enum Instr {
						27 | |     IMov(Val, Val),
						28 | |     IAdd(Val, Val),
						29 | |     ISub(Val, Val),
						...  |
						46 | |     ILabel(i32),
						47 | | }
						   | |_^
						   |
						   = help: remove the prefixes and use full paths to the variants instead of glob imports
						   = help: for further information visit https://rust-lang.github.io/rust-clippy/master/index.html#enum_variant_names
						   = note: `#[warn(clippy::enum_variant_names)]` on by default
					\end{lstlisting}
				\item Labels do not have names that are useful for debugging. They are just \verb|.L<number>| or \verb|.E<number>|. It would be better to have the labels be named after the purpose. 
					For example, if the label is for the start of a loop, it could be named \verb|loop_start<number>| or something similar.
					Labels for error handling should also be named appropriately and stored in a human-readable way instead of just a negative number.
			\end{enumerate}

		\item This snippet handles variable identifiers, including the special \verb|input| identifier.
			\begin{lstlisting}[language=Rust]
				Expr::Id(id) => {
				    if id == "input" {
					instrs.push(Instr::IMov(rax, rdi));
				    } else {
					let id_si = *env.get(id).expect(&format!("Unbound variable identifier {}", id));
					instrs.push(Instr::IMov(rax, stack_slot(id_si)));
				    }
				},
			\end{lstlisting}
	\end{enumerate}

	\pagebreak
	\subsection{Program 2}
	\begin{lstlisting}[language=lisp, title=\texttt{example2.snek}]
		(let ((a 2) (b 3) (c 0) (i 0) (j 0))
		  (loop
		    (if (< i a)
		      (block
			(set! j 0)
			(loop
			  (if (< j b)
			    (block (set! c (sub1 c)) (set! j (add1 j)))
			    (break c)
			  )
			)
			(set! i (add1 i))
		      )
		      (break c)
		    )
		  )
		)
	\end{lstlisting}

	\subsubsection{Relevant snippet 1}

	\begin{lstlisting}[language=Rust]
		Expr::Set(id, e1) => {
		    let id_si = *env.get(id).expect(&format!("Unbound variable identifier {}", id));
		    h_compile_with(e1, instrs, si, env, ls, brkl);
		    instrs.push(Instr::IMov(stack_slot(id_si), rax));
		},
	\end{lstlisting}

	This snippet handles setting a variable to a value. It is called when the \verb|set!| keyword is used.

	\subsubsection{Relevant snippet 2}

	This is the code for compiling a loop.
	\begin{lstlisting}[language=Rust]
		Expr::Loop(e1) => {
		    let labelstart = gen_label(ls);
		    let labelbreak = gen_label(ls);
		    instrs.push(Instr::ILabel(labelstart));
		    h_compile_with(e1, instrs, si, env, ls, Some(labelbreak));
		    instrs.push(Instr::IJmp(labelstart));
		    instrs.push(Instr::ILabel(labelbreak));
		},
	\end{lstlisting}

	\begin{itemize}
		\item Labels don't have useful names, which makes the generated Assembly code a hard to follow.
		\item In the usage of \verb|h_compile_with|, it isn't immediately clear that \verb|instrs| is being added to. 
			Something like \verb|instrs.extend(h_compile_with(e1, si, env, ls, Some(labelbreak)))| would be more explicit and more funcional.
	\end{itemize}

	\subsubsection{Relevant snippet 3}
	
	\begin{lstlisting}[language=Rust, title=Code for compiling \texttt{Expr} into an string of Assembly instructions]
		fn compile(e: &Expr) -> String {
		    let mut instrs = Vec::new();
		    let mut ls = 0;
		    h_compile_with(e, &mut instrs, 2, &HashMap::new(), &mut ls, None);
		    instrs.iter().map(|&i| format!("  {}\n", &instr_to_str(i))).fold(String::new(), |acc, s| acc + &s)
		}
	\end{lstlisting}
	\begin{itemize}
		\item Mutably borrowing a vector of instructions and adding to it is an interesting decision.
		\item Using explicitly called functions to convert Assembly components to strings is a little clumsy when the \verb|Display| exists.
		\item Every line of Assembly is indented by two spaces, including labels. This makes the generated Assembly code a little difficult to read.
		\item What does \verb|h_compile_with| stand for?
	\end{itemize}

	\section{Bugs, Missing Features, and Design Decisions}

	All of my tests passed on this compiler except for a few that checked the error message for when the input is invalid (undefined for Cobra?), and I tried some more tests, but they seem to work.

	\section{Lessons and Advice}

	\begin{enumerate}
		\item Defining common snippets of Assembly as variables and adding them to the vector of instructions is a good idea.
		\item the function \verb|h_compile_with| takes a \verb|&mut Vec<Instr>|.

			Labels don't have useful names

			%Why is \verb|gen_label| a closure?
		\item I will consider using more digits in the numbers for the labels in case there are enough labels that it might make a difference, and separate the digits from the string part to make it clearer.
		\item 
			\begin{itemize}
				\item Run \verb|cargo clippy --all| and fix the problems
				\item Run \verb|cargo fmt| to format the code.
				\item Implement \verb|Display| for \verb|Instr|, \verb|Val|, etc. to get the \verb|ToString| trait so you can do
					\begin{lstlisting}[language=Rust, numbers=none]
        					Instr::IMov(dst, src) => format!("mov {}, {}", dst, src),
					\end{lstlisting}
					to get the string representation of something.
				\item Use different error messages for different errors.
			\end{itemize}

	\end{enumerate}

	%--------------------------------------------------------------------------------------
	
	\chapter*{Compiler 36}

	\section{Tracing the compiler}

	\subsection{Program 1}

	\begin{lstlisting}[language=lisp, title=Factorial]
		(let
			((i 1) (acc 1))
			(loop
				(if (> i input)
					(break acc)
					(block
						(set! acc (* acc i))
						(set! i (+ i 1))
					)
				)
			)
		)
	\end{lstlisting}

	\pagebreak
	\subsubsection{Relevant Snippet 1}

	\begin{lstlisting}[language=Rust, title=from \texttt{src/main.rs}]
		If(cnd, thn, els) => {
		    let else_lbl = make_label("else", &mut mst.label_index);
		    let end_lbl = make_label("endif", &mut mst.label_index);

		    compile_to_instrs(cnd, imst.clone(), mst);
		    mst.instrs.push(ICmp(Reg(RAX), Imm(1))); // only check for false
		    mst.instrs.push(IJmp(Cond::Equal, else_lbl.clone()));
		    compile_to_instrs(thn, imst.clone(), mst);
		    mst.instrs.push(IJmp(Cond::Unconditional, end_lbl.clone())); // jump to end after then
		    mst.instrs.push(ILbl(else_lbl));
		    compile_to_instrs(els, imst, mst);
		    mst.instrs.push(ILbl(end_lbl));
		}
	\end{lstlisting}

	This is the snippet of code that handles \verb|if| expressions. It is difficult to tell what the call to \verb|compile_to_instrs| does at a glance. It looks like it is not doing anything to \verb|instrs|, but it is hidden in \verb|mst|.

	The frequent use of very abbreviated variable names makes it difficult to understand what is going on.

	\pagebreak
	\subsubsection{Relevant Snippet 2}

	\begin{lstlisting}[language=Rust, title=from \texttt{src/main.rs}]
		match op {
		    Plus => {
		        type_check(Type::Number, Reg(RAX), mst);
		        type_check(Type::Number, RegOffset(RSP, imst.stack_index + 1), mst);
		        mst.instrs
			.push(IAdd(Reg(RAX), RegOffset(RSP, imst.stack_index + 1)));
		        overflow_check(mst);
		    }
		    Minus => {
		        type_check(Type::Number, Reg(RAX), mst);
		        type_check(Type::Number, RegOffset(RSP, imst.stack_index + 1), mst);
		        mst.instrs
		            .push(ISub(Reg(RAX), RegOffset(RSP, imst.stack_index + 1)));
		        overflow_check(mst);
		    }
		    Times => {
		        type_check(Type::Number, Reg(RAX), mst);
		        type_check(Type::Number, RegOffset(RSP, imst.stack_index + 1), mst);
		        mst.instrs.extend([
		            ISar(Reg(RAX), Imm(1)), // right shift once before multiplying
		            IMul(Reg(RAX), RegOffset(RSP, imst.stack_index + 1)),
		        ]);
		        overflow_check(mst);
		    }
		    . . .
	\end{lstlisting}

	This snippet of code handles converting the \verb|+|, \verb|-|, and \verb|*| operators from the AST to the corresponding assembly instructions. There is a lot of code reuse in this and the other \verb|BinOp| operators that can be eliminated by moving some of the code to outside the match statement.

	\subsubsection{Relevant Snippet 3}

	\begin{lstlisting}[language=Rust, title=Relevant snippet 3: from \texttt{src/main.rs}, float, floatplacement=H]
        Id(x) if x == "input" => mst.instrs.push(IMov(Reg(RAX), Reg(RDI))),
        Id(x) => match imst.env.get(x) {
            Some(i) => {
                mst.instrs.push(IMov(Reg(RAX), RegOffset(RSP, *i)));
            }
            None => panic!("Unbound variable identifier {x}"),
        },
	\end{lstlisting}

	This snippet of code handles the \verb|input| and variable identifier cases. The \verb|input| case is a special case that is handled separately from the other identifiers. The other identifiers are looked up in the environment and the corresponding value is moved into \verb|RAX|.

	\pagebreak
	\subsection{Program 2}
	\begin{lstlisting}[language=lisp, title=\texttt{example2.snek}]
		(let ((a 2) (b 3) (c 0) (i 0) (j 0))
		  (loop
		    (if (< i a)
		      (block
			(set! j 0)
			(loop
			  (if (< j b)
			    (block (set! c (sub1 c)) (set! j (add1 j)))
			    (break c)
			  )
			)
			(set! i (add1 i))
		      )
		      (break c)
		    )
		  )
		)
	\end{lstlisting}

	\subsubsection{Relevant Snippet 1}

	\begin{lstlisting}[language=Rust, title=from \texttt{src/main.rs}]
		Set(x, e) => match &imst.env.get(x) {
		    Some(&i) => {
			compile_to_instrs(e, imst, mst);
			mst.instrs.push(IMov(RegOffset(RSP, i), Reg(RAX)));
		    }
		    None => panic!("Unbound variable identifier {x}"),
		},
	\end{lstlisting}

	\subsubsection{Relevant Snippet 2}

	\begin{lstlisting}[language=Rust]
        Let(binds, e) => {
            if binds
                .iter()
                .map(|(s, _)| s.clone())
                .collect::<im::HashSet<String>>()
                .len()
                != binds.len()
            {
                panic!("Duplicate binding")
            }
	    . . .
	\end{lstlisting}

	This snippet is where the \verb|let| bindings are checked for duplicates. Using a HashSet is clever, and better than what I did.

	\subsubsection{Relevant Snippet 3}

	\begin{lstlisting}[language=Rust]
macro_rules! gen_asm {
    ($ops:ident, $x:ident, $y:ident, $instr:ident) => {
        match ($x, $y) {
            (Val::Reg(r), Val::Imm(i)) => {
                dynasm!($ops ; .arch x64 ; $instr Rq(reg_to_idx(r)), *i as _);
            }
            (Val::Reg(r0), Val::Reg(r1)) => {
                dynasm!($ops ; .arch x64 ; $instr Rq(reg_to_idx(r0)), Rq(reg_to_idx(r1)));
            }
            (Val::Reg(r0), Val::RegOffset(r1, i)) => {
                dynasm!($ops ; .arch x64 ; $instr Rq(reg_to_idx(r0)), [Rq(reg_to_idx(r1)) - (*i as i32) * 8]);
            }
            (Val::RegOffset(r0, i), Val::Reg(r1)) => {
                dynasm!($ops ; .arch x64 ; $instr [Rq(reg_to_idx(r0)) - (*i as i32) * 8], Rq(reg_to_idx(r1)));
            }
            _ => panic!("cannot match with generated asm")
        }
    };
}

pub fn instr_to_asm(
    i: &Instr,
    lbls: &im::HashMap<&String, DynamicLabel>,
    ops: &mut dynasmrt::x64::Assembler,
) {
    match i {
        Instr::IMov(Val::Reg(r), Val::Imm(i)) => {
            dynasm!(ops ; .arch x64 ; mov Rq(reg_to_idx(r)), QWORD *i);
        }
        Instr::IMov(x, y) => gen_asm!(ops, x, y, mov),
        Instr::IAdd(x, y) => gen_asm!(ops, x, y, add),
        Instr::ISub(x, y) => gen_asm!(ops, x, y, sub),
        Instr::ICmp(x, y) => gen_asm!(ops, x, y, cmp),
        Instr::ITest(x, y) => gen_asm!(ops, x, y, test),
        Instr::IMul(Val::Reg(r), Val::Imm(i)) => {
	. . .
	\end{lstlisting}

	This snippet is where the \verb|Instr| enums are converted to assembly. A macro is used to generate the assembly for each instruction. I don't understand this, but I will look into it.

	\section{Bugs, Missing Features, and Design Decisions}

	Missing feature: checking \verb|input| for overflow and nice error messages for other invalid inputs.

	\begin{lstlisting}[language=bash, numbers=none]
		$ cat tests/input.snek
		input
		$ make tests/input.run && tests/input.run 4611686018427387904
		make: 'tests/input.run' is up to date.
		-4611686018427387904
	\end{lstlisting}

	Relevant snippet of code:

	\begin{lstlisting}[language=Rust, title=\texttt{runtime/start.rs}]
		fn parse_input(input: &str) -> u64 {
		    match input {
			"true" => 3,
			"false" => 1,
			_ => (input.parse::<i64>().unwrap() as u64) << 1, // TODO: check for overflow
		    }
		}
	\end{lstlisting}

	\begin{itemize}
		\item For checking overflow, you can shift the number left by 1 bit and then shift it back right by 1 bit. If the number is the same, then it didn't overflow. If it's different, then it overflowed.

			Or you can extract the two most significant bits and check if they are the same. If they are, then it didn't overflow. If they are different, then it overflowed.

		\item For the error messages, you can use \verb|unwrap_or_else| to panic with a custom error message if the \verb|Option| is \verb|None|.
	\end{itemize}

	\section{Lessons and Advice}

	\begin{enumerate}
		\item The mutable and immutable states are stored in structs that are passed to the \verb|compile_to_instrs| function. This should mean the function signature and every call to the function does not have to change if a new field is added to the structs.
		\item Hiding \verb|instrs| in the \verb|MutState| struct
		\item Improvements to mine
			\begin{itemize}
				\item Look into \verb|unwrap_or|?
				\item \verb|dynasm| stuff
				\item Look into macros
			\end{itemize}
		\item Improvements to author of compiler
			\begin{itemize}
				\item Run \verb|cargo clippy --all| and fix the problems
				\item Use more explicit and descriptive names for variables, enum variants, etc. Don't make people decipher what \verb|mst| or \verb|Ilbl| means.
					\begin{itemize}
						\item \verb|type_check| vs \verb|type_test|?
					\end{itemize}
				\item No \verb|I| prefix in \verb|Instr| enum variants
				\item Organize files better. Keep things that do one thing in one file, instead of putting enums in \verb|lib.rs| and using them in \verb|main.rs|, and maybe more files?
			\end{itemize}
	\end{enumerate}

\end{document}
